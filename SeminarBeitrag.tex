% !TEX encoding = UTF-8 Unicode

\documentclass[ngerman]{seminarbeitrag} % im Format "seminarvorlage" mit Deutsch in neuer Rechtschreibung

\usepackage[utf8]{inputenc} % Kodierung der Non-ASCII-Zeichen

\begin{document}

% Unbedingt notwendig: Titel, Autoren
\title{Ein intelligentes öffentliches Transportsystem basierend auf IoT}
\author{Luca Bernstein\and\ Mathis Neunzig}

\maketitle % Titelangaben produzieren, kein Inhaltsverzeichnis

% TODO: Abstract mit neuem Inhalt wieder einführen:
\begin{abstract}
TODO werden verschiedene Ansätze zur Optimierung öffentlicher Transportsysteme mithilfe von IoT beleuchtet.

\keywords{TODO, IoT.}% Keywords innerhalb vom Abstract
\end{abstract}

% ---------------
\section{Einleitung}
% Begründung, Ziel, Abgrenzung des Themas.
\subsection{Problem-/Fragestellung}
\subsection{Themenüberblick}
\section{Grundlagen}
\subsection{Definitionen, Begriffe und Konzepte}
\section{Architektur und Technologien öffentlicher Verkehrssysteme}
\section{Ansätze zur Effizienz- und Effektivitätssteigerung}
\subsubsection{Dynamische Steuerung und Anpassungen des Verkehrsbetriebs}
\subsubsection{Mathematisches Modell zur Betriebsoptimierung}
\section{Ansätze zur Effizienz- und Effektivitätssteigerung}
\section{Zusammenfassung und Fazit}
% Erkenntnisse und Ergebnisse. Antwort auf gestellte Frage.
% Ausblick?
\newpage
% ---------------

% Section-Überschriften werden automatisch in GROSSBUCHSTABEN gesetzt
% \section{Einleitung}\label{einleitung}

Die Geschichte der Gummi\-bär\-chen ist voller                       % \- Trennhilfe 
Über\-raschun\-gen. Schon 1922 wurden diese Tierchen erstmals\ldots\ % \ldots macht ...
Siehe insbesondere die Standardwerke von Jung und Alt~\cite{DEMO9999,DEMO9999}.

Zwei Jahre nach der Firmengründung erfand
Hans Riegel den Vorläufer der Goldbären, damals noch „Tanz\-bären“ genannt.
Diese waren allerdings nicht nur größer als die heutigen Gummibärchen,
sondern auch weicher, da zu seiner Herstellung statt der heute üblichen
Gelatine noch \emph{Gummi arabicum} verwendet wurde.
Inzwischen werden Gummibären in vielen Variationen von anderen Herstellern angeboten~\cite{DEMO9999}.

1925 begann Haribo mit der Herstellung von Lakritz\-produkten.
Anfang der 1930er Jahre entstanden die Vertriebsorganisation in Deutschland
und der Hauptbau der neuen Fabrikationsanlage. 1935 wurde in Kopenhagen\ldots

% Jede Section am besten mit einem Kommentar hier im Quelltext markieren
% \section{Und Erwachsene ebenso?}

Das ist die Frage, die seit dem
21.~Jahrhundert aufgeworfen wurde~\cite[S.~237f]{DEMO9999}.

% \subsection{Konsumentenalter}\label{alt}
Meist wurde implizit angenommen, dass Kinder die Hauptzielgruppe sind.
Diese Auffassung lässt sich allerdings nicht mehr aufrecht halten~\cite{DEMO9999}.

% \subsection{Varianten}\label{var}
Weitere Varianten sind Gummibären auf Fruchtsaftbasis (mit dem Zusatz „Frucht“ deklariert), die Haribo 2009 auf den Markt brachte. Fruchtsaft wird bereits seit 1989 auch zum Färben verwendet.
In \cref{niko} sehen wir ein Beispiel.

\begin{figure}
\begin{center}
\unitlength8mm % hier skalieren, falls gewünscht
\begin{picture}(4,6)
%außen
\put(0,0){\line(1,0){4}}
\put(4,0){\line(0,1){4}}
\put(4,4){\line(-1,1){2}}
\put(2,6){\line(-1,-1){2}}
\put(0,4){\line(0,-1){4}}
%innen
\put(0,0){\line(1,1){4}}
\put(4,4){\line(-1,0){4}}
\put(0,4){\line(1,-1){4}}
\end{picture}
\end{center}
\caption{Das Haus des Nikolaus als Graph in seiner ersten, ursprünglichen Form,
         siehe auch~\protect\cite[S.~93]{DEMO9999}. Erwähnenswert ist insbesondere,
         dass die Figur gezeichnet werden kann, ohne den Stift abzusetzen, wenn man an
         der richtigen Stelle (wo?) beginnt.}
\label{niko}
\end{figure}

% Eine neue Seite anfangen mit
%\pagebreak

In \cref{tttabelle} ist das Ganze tabellarisch dargestellt.


\begin{table}
\begin{center}
\begin{tabular}{|c|c|c|}
\hline
Material & Tierart & essbar\\
\hline
Gummi & Bär & ja\\
\hline
\end{tabular}
\end{center}
\caption{Eine Übersicht zu den Fachbegriffen.}
\label{tttabelle}
\end{table}

% und schon der letzte Abschnitt
% \section{Zusammenfassung}

In dieser Abhandlung wurde die Geschichte neu interpretiert.
Es ergaben sich völlig neuartige Forschungsansätze, dargestellt in \cref{alt} und~\ref{var},
die so vielfältig sind, dass sich die Auswirkungen gegen\-wärtig %Trennhilfe \-
kaum abschätzen lassen. Auch Artikel mit vielen Autoren~\cite{DEMO9999}
befassen sich mit diesem Thema.


\bibliographystyle{abbrv}
\bibliography{literatur} % Daten aus der Datei literatur.bib verwenden.

\end{document}
