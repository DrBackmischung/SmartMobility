% !TEX encoding = UTF-8 Unicode

% Beispiel für ein LaTeX-Dokument im Format "seminarvorlage"
\documentclass[ngerman]{seminarbeitrag}
% ngerman = Deutsch in neuer Rechtschreibung
% english = Englisch

\usepackage[utf8]{inputenc} % Kodierung der Non-ASCII-Zeichen

\begin{document}

% Unbedingt notwendig: Titel, Autoren
\title{Ein intelligentes öffentliches Transportsystem basierend auf IoT}
\author{Luca Bernstein\and\ Mathis Neunzig}

\maketitle% Titelangaben produzieren, kein Inhaltsverzeichnis

\begin{abstract}
Gummibärchen (auch Gummibären bzw. Goldbären) sind Fruchtgummis in Form von etwa
zwei Zentimeter großen, stilisierten Bären. Sie werden in unterschiedlichen Farben hergestellt
und bestehen im Wesentlichen aus Zucker, Zuckersirup und einer erstarrten
Gelatine-Mischung, die ihnen ihre gummiartige Konsistenz verleiht.

\keywords{Süßigkeiten, Gelatine, Bär, Lebensmittelfarbe.}% Keywords innerhalb vom Abstract
\end{abstract}

% Section-Überschriften werden automatisch in GROSSBUCHSTABEN gesetzt
\section{Einleitung}\label{einleitung}

Die Geschichte der Gummi\-bär\-chen ist voller                       % \- Trennhilfe 
Über\-raschun\-gen. Schon 1922 wurden diese Tierchen erstmals\ldots\ % \ldots macht ...
Siehe insbesondere die Standardwerke von Jung und Alt~\cite{ACM2019,Ivory2001}.

Zwei Jahre nach der Firmengründung erfand
Hans Riegel den Vorläufer der Goldbären, damals noch „Tanz\-bären“ genannt.
Diese waren allerdings nicht nur größer als die heutigen Gummibärchen,
sondern auch weicher, da zu seiner Herstellung statt der heute üblichen
Gelatine noch \emph{Gummi arabicum} verwendet wurde.
Inzwischen werden Gummibären in vielen Variationen von anderen Herstellern angeboten~\cite{gummi}.

1925 begann Haribo mit der Herstellung von Lakritz\-produkten.
Anfang der 1930er Jahre entstanden die Vertriebsorganisation in Deutschland
und der Hauptbau der neuen Fabrikationsanlage. 1935 wurde in Kopenhagen\ldots

% Jede Section am besten mit einem Kommentar hier im Quelltext markieren
\section{Und Erwachsene ebenso?}

Das ist die Frage, die seit dem
21.~Jahrhundert aufgeworfen wurde~\cite[S.~237f]{Ivory2001}.

\subsection{Konsumentenalter}\label{alt}
Meist wurde implizit angenommen, dass Kinder die Hauptzielgruppe sind.
Diese Auffassung lässt sich allerdings nicht mehr aufrecht halten~\cite{Black1988}.

\subsection{Varianten}\label{var}
Weitere Varianten sind Gummibären auf Fruchtsaftbasis (mit dem Zusatz „Frucht“ deklariert), die Haribo 2009 auf den Markt brachte. Fruchtsaft wird bereits seit 1989 auch zum Färben verwendet.
In \cref{niko} sehen wir ein Beispiel.

\begin{figure}
\begin{center}
\unitlength8mm % hier skalieren, falls gewünscht
\begin{picture}(4,6)
%außen
\put(0,0){\line(1,0){4}}
\put(4,0){\line(0,1){4}}
\put(4,4){\line(-1,1){2}}
\put(2,6){\line(-1,-1){2}}
\put(0,4){\line(0,-1){4}}
%innen
\put(0,0){\line(1,1){4}}
\put(4,4){\line(-1,0){4}}
\put(0,4){\line(1,-1){4}}
\end{picture}
\end{center}
\caption{Das Haus des Nikolaus als Graph in seiner ersten, ursprünglichen Form,
         siehe auch~\protect\cite[S.~93]{Ivory2001}. Erwähnenswert ist insbesondere,
         dass die Figur gezeichnet werden kann, ohne den Stift abzusetzen, wenn man an
         der richtigen Stelle (wo?) beginnt.}
\label{niko}
\end{figure}

% Eine neue Seite anfangen mit
%\pagebreak

In \cref{tttabelle} ist das Ganze tabellarisch dargestellt.


\begin{table}
\begin{center}
\begin{tabular}{|c|c|c|}
\hline
Material & Tierart & essbar\\
\hline
Gummi & Bär & ja\\
\hline
\end{tabular}
\end{center}
\caption{Eine Übersicht zu den Fachbegriffen.}
\label{tttabelle}
\end{table}

% und schon der letzte Abschnitt
\section{Zusammenfassung}

In dieser Abhandlung wurde die Geschichte neu interpretiert.
Es ergaben sich völlig neuartige Forschungsansätze, dargestellt in \cref{alt} und~\ref{var},
die so vielfältig sind, dass sich die Auswirkungen gegen\-wärtig %Trennhilfe \-
kaum abschätzen lassen. Auch Artikel mit vielen Autoren~\cite{Black1988}
befassen sich mit diesem Thema.


% Bibliographie entweder direkt hier eingeben (nur im Notfall)...
%\begin{thebibliography}{9}
%\bibitem{ACM2019}
%ACM.
%\newblock How to classify works using ACM's computing classification system.
%\newblock \url{http://www.acm.org/class/how_to_use.html}.
%
%\bibitem{Ivory2001}
%M.~Y. Ivory and M.~A. Hearst.
%\newblock The state of the art in automating usability evaluation of user
%  interfaces.
%\newblock {\em ACM Comput. Surv.}, 33(4):470--516, 2001.
%
%\end{thebibliography}

% ... oder die Bibliographie mit Hilfe von BibTeX generieren,
% dies ist auf jeden Fall die bessere Lösung und sollte nach
% Möglichkeit immer verwendet werden:
\bibliographystyle{abbrv}
\bibliography{literatur} % Daten aus der Datei literatur.bib verwenden.

\end{document}
