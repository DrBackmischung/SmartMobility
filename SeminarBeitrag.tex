% !TEX encoding = UTF-8 Unicode

\documentclass[ngerman]{seminarbeitrag} % im Format "seminarvorlage" mit Deutsch in neuer Rechtschreibung

\usepackage[utf8]{inputenc} % Kodierung der Non-ASCII-Zeichen

\begin{document}

% Unbedingt notwendig: Titel, Autoren
\title{Ein intelligentes öffentliches Transportsystem basierend auf IoT}
\author{Luca Bernstein\and\ Mathis Neunzig}

\maketitle % Titelangaben produzieren, kein Inhaltsverzeichnis

% --- ONLY FOR DISPLAY PURPOSES ---
\setcounter{tocdepth}{2}
\tableofcontents{}
\newpage
% ---------------------------------

% TODO: Abstract mit neuem Inhalt wieder einführen:
\begin{abstract}
TODO werden verschiedene Ansätze zur Optimierung öffentlicher Transportsysteme mithilfe von IoT beleuchtet.

\keywords{TODO, IoT.}% Keywords innerhalb vom Abstract
\end{abstract}

%=================================================================================
\pagebreak
\section{Einleitung}
\subsection{Motivation}
\subsubsection{Urbanisierung}
\subsubsection{Verkehrsprobleme}
\subsubsection{Umwelt}
\subsection{Zielsetzung}
\subsection{Struktur der Arbeit}

%=================================================================================
\pagebreak
\section{Grundlagen}
\subsection{Begriffserklärungen}
\subsubsection{Internet of Things (Gubbi et al)}
\subsubsection{Intelligent Transportation Systems (ITS) (Elkosantini et al)}
\subsubsection{Intelligent Public Transportation Systems (IPTS) (Elkosantini et al)}
\subsection{Architekturen}
\subsubsection{Sensorik}
\subsubsection{Kommunikation}
\subsubsection{Cloud-Integration}
\subsubsection{3-Layer-Architektur (Luo et al.)}

%=================================================================================
\pagebreak
\section{Architektur und Technologien intelligenter öffentlicher Verkehrssysteme}
\subsection{Erfassung und Analyse von Echtzeitdaten}
\subsubsection{Fahrzeug-Tracking (GPS, AVLS)} 
\subsubsection{Fahrgastzählung (APC, z.B. Infrarot-Technologie bei Li, Systeme wie SUICA, PASMO, Oyster, etc.)}
\subsection{Kommunikationstechnologien}
\subsubsection{Mobilfunk (LTE, GPRS)}
\subsubsection{WLAN}
\subsubsection{RFID}
\subsection{Entscheidungsunterstützung}
\subsubsection{Decision Support Systems (DSS) für Disposition und Fahrplanoptimierung}

%=================================================================================
\pagebreak
\section{Ansätze zur Effizienz- und Effektivitätssteigerung}
\subsection{Effizienzanalysen im öffentlichen Verkehr (Daraio et al.)}
\subsubsection{Ökonomische Indikatoren und Umweltziele}
\subsubsection{Datenbasierte Optimierung: DEA, SFA, Translog-Modelle}
\subsection{Flussprognosen und dynamische Fahrplansteuerung}
\subsubsection{Rolling Horizon Optimization (Luo et al.)}
\subsubsection{Pattern Mining aus Verkehrsdaten}
\subsection{Erweiterte Entwicklungen: Vernetzte und autonome Fahrzeuge (CAVs)}
\subsubsection{Rolle von Connected and Automated Vehicles (nach Guo et al. )}
\subsubsection{Möglichkeiten durch V2V- und V2I-Kommunikation}
\subsubsection{Verbesserte Ampelsteuerung und Fahrpläne}
\subsubsection{Herausforderungen bei der Integration von CAVs in bestehende Systeme}

%=================================================================================
\pagebreak
\section{Herausforderungen und offene Forschungsfragen}

\subsection{Technische Herausforderungen}
\subsubsection{Datensicherheit und Datenschutz}
\subsubsection{Skalierung und Standardisierung}
\subsection{Organisatorische und gesellschaftliche Herausforderungen}
\subsubsection{Finanzierung}
\subsubsection{Nutzerakzeptanz}
\subsection{Offene Forschungsthemen}
\subsubsection{Robustere KI-Modelle für Verkehrssteuerung}
\subsubsection{Integration dezentraler Sensornetze}

%=================================================================================
\pagebreak
\section{Fazit}
\subsection{Zusammenfassung der wichtigsten Erkenntnisse}
\subsection{Bedeutung intelligenter Transportsysteme für nachhaltige Mobilität}

\bibliographystyle{abbrv}
\bibliography{literatur} % Daten aus der Datei literatur.bib verwenden.

\end{document}
